% for notes environment
\usepackage{xsavebox}

\usepackage{graphicx}
\usepackage{luatexja}
\usepackage[hiragino-pron, nfssonly, deluxe, expert]{luatexja-preset}
\usepackage{fontspec}
\usepackage{epigraph}
\usepackage{etoolbox}
\usepackage{tikz}
\usepackage{framed}
\usepackage{mathtools}
\usepackage{listings}
\usepackage{libertine}
\usepackage[libertine]{newtxmath}
\usepackage{bxcoloremoji}
\usepackage{xcolor}
\usepackage{diagbox}
\usepackage{caption}
\usepackage{appendixnumberbeamer}
\usepackage{plantuml}

\usetikzlibrary{fit}
\usetikzlibrary{shapes.misc}
\usetikzlibrary{calc,positioning}

\setmonofont{CMU Typewriter Text}

\definecolor{links}{HTML}{2A1B81}
\hypersetup{colorlinks,linkcolor=,urlcolor=links}

\usetheme{Boadilla}
\usecolortheme{seahorse}
% \usefonttheme{serif}

\setbeamercolor{page number in head/foot}{bg=blue!10}
\setbeamertemplate{footline}{%
  \leavevmode%
  \hbox{%
    \begin{beamercolorbox}[wd=.35\paperwidth,ht=2.25ex,dp=1ex,center]{author in head/foot}%
      \usebeamerfont{author in head/foot}\insertshortauthor\hspace*{1ex}(\insertshortinstitute)
    \end{beamercolorbox}%
    \begin{beamercolorbox}[wd=.3\paperwidth,ht=2.25ex,dp=1ex,center]{title in head/foot}%
      \usebeamerfont{title in head/foot}\insertshorttitle
    \end{beamercolorbox}%
    \begin{beamercolorbox}[wd=.25\paperwidth,ht=2.25ex,dp=1ex,center]{date in head/foot}%
      \insertshortdate{} @ \InsertConference
    \end{beamercolorbox}%
    \begin{beamercolorbox}[wd=.1\paperwidth,ht=2.25ex,dp=1ex,center]{page number in head/foot}%
      \insertframenumber{} / \inserttotalframenumber\hspace*{1ex}
    \end{beamercolorbox}}%
  \vskip0pt%
}

\beamertemplatenavigationsymbolsempty

\setbeamertemplate{bibliography item}{\insertbiblabel}
\setbeamersize{description width=1cm}
\setbeamertemplate{items}[circle]
\setbeamertemplate{section in toc}[circle]
\setbeamertemplate{subsection in toc}{%
  \leavevmode\leftskip=2em
  {%
    \usebeamerfont*{itemize item}%
    \usebeamercolor{subsection number projected}%
    \color{bg}%
    \raise1.25pt\hbox{\donotcoloroutermaths$\bullet$}}%
  \hskip1.5ex\inserttocsubsection\par}

% Definitions for the title page
\newcommand*{\GitHub}[1]{%
  \gdef\InsertGitHub{#1}%
}
\newcommand*{\Email}[1]{%
  \gdef\InsertEmail{\href{mailto:#1}{#1}}%
}
\newcommand*{\Conference}[1]{%
  \gdef\InsertConference{#1}%
}
\setbeamerfont{title}{size=\huge, series=\bfseries, family=\mcfamily\rmfamily}
\setbeamercolor{title}{bg=white}
\setbeamerfont{subtitle}{size=\Large, series=\mdseries, family=\gtfamily\sffamily}
\setbeamerfont{email}{size=\scriptsize, family=\ttfamily}
\setbeamercolor{email}{bg=white}
\setbeamerfont{date}{family=\gtfamily\sffamily}
\setbeamerfont{vc}{size=\scriptsize, family=\ttfamily}
\setbeamercolor{vc}{bg=white}

\renewcommand{\figurename}{Fig}

\input{vc.tex}

\setbeamertemplate{title page}
{%
  \vbox{}
  \vfill
  \begingroup
    \centering
    \hrulefill\par%
    \vskip1ex\par%
    \begin{beamercolorbox}[sep=0pt,center,shadow=false,rounded=true]{title}
      \vfill
      \usebeamerfont{title}\inserttitle\par%
      \ifx\insertsubtitle\@empty%
      \else%
        \vskip0.5ex%
        {\usebeamerfont{subtitle}\usebeamercolor[fg]{subtitle}\insertsubtitle\par}%
      \fi%
      \vfill  
    \end{beamercolorbox}%
    \hrulefill\par%
    \vskip2ex%
    \begin{beamercolorbox}[sep=0pt,center,shadow=false,rounded=true]{author}
      \usebeamerfont{author}\insertauthor
    \end{beamercolorbox}
    \begin{beamercolorbox}[sep=0pt,center,shadow=false,rounded=true]{email}
      \usebeamerfont{email}\InsertEmail
    \end{beamercolorbox}
    \vskip0.1ex
    \begin{beamercolorbox}[sep=5pt,center,shadow=false,rounded=true]{institute}
      \usebeamerfont{institute}\insertinstitute
    \end{beamercolorbox}
    \begin{beamercolorbox}[sep=5pt,center,shadow=false,rounded=true]{date}
      \usebeamerfont{date}\insertdate @ \InsertConference
    \end{beamercolorbox}
    \begin{beamercolorbox}[sep=0pt,center,shadow=false,rounded=true]{vc}
      \usebeamerfont{vc}
      \url{https://github.com/\InsertGitHub} (\texttt{\GITAbrHash})
    \end{beamercolorbox}
    % {\centering
    %   \href{https://creativecommons.org/licenses/by-nc/4.0/}{%
    %     \includegraphics[width=0.1\textwidth]{img/by-nc.pdf}%
    %   }%
    % }
    {\usebeamercolor[fg]{titlegraphic}\inserttitlegraphic\par}
  \endgroup
  \vfill
}
\setbeamertemplate{blocks}[rounded][shadow=false]

% ============ ここを消すとNote消える ================
% \mode<handout>{%
%   \usepackage{pgfpages}
%   \setbeameroption{show notes on second screen=right}
%   \setbeamertemplate{note page}{%
%     \vspace{2ex}\insertnote%
%   }
% }
% ============ ここを消すとNote消える ================


\renewcommand{\kanjifamilydefault}{\gtdefault}

\resetcounteronoverlays{lstlisting}
\definecolor{bluegray}{rgb}{0.4, 0.6, 0.8}
\DeclareCaptionFormat{listing}{{\color{bluegray}\lstlistingname}#2#3}
\captionsetup[lstlisting]{format=listing, font={footnotesize}}

\setmonofont[Ligatures=TeX]{CMU Typewriter Text}

\setbeamertemplate{items}[circle]

\input{./lib/quotebox.tex}
\input{./lib/footnotemark.tex}
\input{./lib/ballon.tex}
\input{./lib/callout.tex}
\input{./lib/listings.tex}
\input{./lib/notes.tex}
\input{./lib/stack.tex}

\newcommand\ce[1]{%
  \coloremojiucs{#1}
}

\presetkeys{todonotes}{inline, noinlinepar}{}

\title[Webセキュリティー入門]{%
  Webセキュリティー入門%
}
\subtitle{CSRFとCORS}
\author[吉村 優]{%
  吉村 優(\textsc{Yoshimura} Hikaru)
}
\Email{hikaru\_yoshimura@r.recruit.co.jp}
\date[July 15, 2021]{%
  {\rmfamily\itshape \oldstylenums{July 15, 2021}}
}
\Conference{Webセキュリティー勉強会}
\institute[\InsertEmail]{%
  株式会社リクルート\\
  Recruit Co., Ltd
}
\GitHub{y-yu}

\begin{document}

\setlength{\leftmargini}{1em}

\begin{frame}
  \maketitle
\end{frame}

\begin{frame}
  \frametitle{Table of contents}

  \tableofcontents
\end{frame}

\section{自己紹介}
\begin{frame}
  \frametitle{自己紹介}
  
  \begin{columns}
    \begin{column}{0.3\textwidth}
      \begin{center}
        \begin{figure}[h]
          \includegraphics[width=0.95\textwidth]{img/bird.png}%
        \end{figure}
      \end{center}
 
      \begin{table}[h]
        \begin{tabular}{ll}
          Slack & \texttt{@y-yu} \\
          Twitter & \href{https://twitter.com/\_yyu\_}{@\_yyu\_} \\
          Qiita &  \href{https://qiita.com/yyu}{yyu} \\
          GitHub &  \href{https://github.com/y-yu}{y-yu}
        \end{tabular}
      \end{table}
    \end{column}
    \begin{column}{0.7\textwidth}
      \begin{itemize}
        \item 株式会社ドワンゴ(新卒)
        \item 株式会社リクルート(中途)

        \item セキュリティー・暗号
        \item CTF(\url{https://urandom.team/})
        \begin{itemize}
          \item SECCON 2014 オンライン予選 優勝
          \item SECCON 2015 x CEDEC CHALLENGE ゲームクラッキング&チートチャレンジ 優勝
          \item IWSEC Cup 2015 Gold Prize
          \item サイバーコロッセオ×SECCON 2016 準優勝
          \item SECCON CTF 2018 International 3rd place
          \item SECCON CTF Beginners 2020 29th
        \end{itemize}
      \end{itemize}
    \end{column}
  \end{columns}
\end{frame}

\section{CSRFとは?}

\begin{frame}[fragile]
  \frametitle{CSRFとは?}

  \begin{itemize}
    \item 攻撃者がユーザーの意図しないリクエストを送信させる攻撃

    \pause
    \item 典型的にはこんな感じ\ce{:point_down:}
    \begin{minipage}{.8\textwidth}
      \begin{plantuml}
        @startuml
        autonumber
        actor Attacker AS "攻撃者" #dc322f
        actor User As "ユーザー"
        participant AttackWebsite AS "罠サイト" #dc322f
        database Server AS "ターゲットサーバー"
       
        Attacker -> User: 罠サイトへ誘導
        User -> AttackWebsite: 罠サイトのJSやimgタグ起動
        AttackWebsite -> Server: HTTPリクエスト送信

        @enduml
      \end{plantuml}
    \end{minipage}

    \item 今日はこのCSRFと、それを\textbf{クロスサイト通信}は許可しつつ防御する話
  \end{itemize}
\end{frame}

\begin{frame}[fragile]
  \frametitle{CSRFとは?}

  \begin{columns}
    \begin{column}{.4\textwidth}
      \begin{minipage}{\textwidth}
        \begin{plantuml}
          @startuml
          
          participant AttackWebsite AS "罠サイト" #dc322f
          database Server AS "ターゲットサーバー"
       
          autonumber 4
          
          AttackWebsite -> Server: HTTPリクエスト送信
       
          @enduml
        \end{plantuml}
      \end{minipage}
    \end{column}
    \begin{column}{.6\textwidth}
      \simplecallout[{\includegraphics[width=1.3cm]{./img/computer_man.png}}]{-}{orange!10}{なぜ攻撃者は罠サイトを使うのか?\ce{:thinking:}}

      \pause
      \simplecallout[{\includegraphics[width=1.1cm]{./img/hacker_laugh.png}}]{+}{green!10}{%
        JSを動かしたいけど正規のページに \\
        \lstinline|<script>|とかを入れられないから%
      }

      \begin{itemize}
        \item 正規のページにJSを無理やり入れる攻撃はXSSとなる
      \end{itemize}
    \end{column}
  \end{columns}
\end{frame}

\begin{frame}[fragile]
  \frametitle{CSRF \textit{vs} XSS}
  
  \begin{itemize}
    \item ターゲットのWebサイトで直接攻撃用のJSを埋め込めれば(XSS)それが早い

    \tikz[baseline,remember picture]{\node[anchor=base] (a){};}\hfill\tikz[baseline,remember picture]{\node[anchor=base] (b){};}
    \begin{center} 
      \begin{minipage}{.6\textwidth}
        \begin{plantuml}
          @startuml
       
          actor Attacker AS "攻撃者" #dc322f
          actor User AS "ユーザー"
          participant Website AS "ターゲットのWebサイト"
          database Server AS "ターゲットサーバー"
       
          Attacker -> Website: 攻撃用のJSを仕込む
          User -> Website: Webサイトにやってくる
          Website -> Website: 攻撃用JSが起動
          Website -> Server: 意図しないHTTPリクエスト送信
          @enduml
        \end{plantuml}
      \end{minipage}
    \end{center}
    \tikz[baseline,remember picture]{\node[anchor=base] (c){};}\hfill\tikz[baseline,remember picture]{\node[anchor=base] (d){};}
    
    \pause
    \item しかしそれができないので、攻撃者は罠サイトを利用する
    \begin{tikzpicture}[remember picture, overlay]
      \draw[red!70, line width=1cm] (a) -- ($(d) + (0,1)$) (b) -- ( $(c) + (0,1)$ );
    \end{tikzpicture}
  \end{itemize}
\end{frame}

\section{Same Origin Policy(SOP)とCross-Site Request}

\begin{frame}[fragile]
  \frametitle{CSRFトークン}

  \begin{itemize}
    \item \textbf{CSRFトークン}を使えばどうだろうか?
  \end{itemize}

  \begin{columns}
    \begin{column}{.35\textwidth}
      \begin{minipage}{.85\textwidth}
        \begin{plantuml}
          @startuml
          
          participant Website AS "Webサイト"
          database Server AS "サーバー"
       
          autonumber
          Website -> Server: CookieとCSRFトークン
          Server -> Server: 処理
          @enduml
        \end{plantuml}
      \end{minipage}

      \pause
      \begin{minipage}{\textwidth}
        \begin{plantuml}
          @startuml
          
          participant AttackWebsite AS "罠サイト" #dc322f
          database Server AS "サーバー"
          
          AttackWebsite -> Server: HTTPリクエスト送信
          note right
            区別
            できない!
          end note
          @enduml
        \end{plantuml}
      \end{minipage}
    \end{column}
    \begin{column}{.65\textwidth}
      \simplecallout[{\includegraphics[width=1.3cm]{./img/hacker_white_angry.png}}]{-}{orange!10}{%
        Webサイトとサーバーは別ドメインなので、\\
        罠サイトと区別できない\ce{:innocent:}%
      }
       
      \pause
      \simplecallout[{\includegraphics[width=1.3cm]{./img/chrome.png}}]{+}{green!10}{%
        クロスサイトの通信を基本的に禁止とするか \\
        (過激派)%
      }
    \end{column}
  \end{columns}
\end{frame}

\begin{frame}[fragile]
  \frametitle{Same Origin Policy(SOP)}

  \pause
  \begin{itemize}
    \item Webブラウザーはこのような罠サイトを跨いだHTTPリクエストが
    攻撃に発展すると考えて、\emph{Same Origin Policy(SOP)}を持っている
   
    \item SOPによってオリジンに跨がる\emph{Cross-site Request}を基本的にはできない
  \end{itemize}

  \begin{columns}
    \begin{column}{.5\textwidth}
      \begin{minipage}{.9\textwidth}
        \begin{plantuml}
          @startuml
          
          participant AttackWebsite AS "罠サイト" #dc322f
          database Server AS "ターゲットサーバー"
       
          autonumber 4
          
          AttackWebsite ->x Server: HTTPリクエスト送信
          note right
            SOPによりサイトを
            跨いだ通信はブロック!
          end note
       
          @enduml
        \end{plantuml}
      \end{minipage}
    \end{column}
    \begin{column}{.5\textwidth}        
      \begin{itemize}
        \item ただし\lstinline|<img>|タグとかで画像を読みこんだりするのはクロスサイトでもできる
       
        \item GETで副作用が起きてしまうのはSOPの恩恵が受けられず、
        CSRFに発展する危険性がある
      \end{itemize}
    \end{column}
  \end{columns}
  
  \pause
  \simplecallout[{\includegraphics[width=1.3cm]{./img/computer_programming_man.png}}]{+}{red!10}{%
    とはいえこれでは不便!無理やり回避するか!%
  }
\end{frame}

\begin{frame}[fragile]
  \frametitle{JSONP(JSON with Padding)}
  
  \begin{itemize}    
    \item (他にもいろいろあるかもしれないけど)SOPを回避する\textbf{かつてあった}代表的な方法が
    \emph{JSONP(JSON with Padding)}である

    \pause
    \begin{columns}
      \begin{column}{.5\textwidth}
\begin{lstlisting}[style=scala, caption={サーバー}]
val json = {'json': ...} /* 渡したいJSON! */
val callback = request.query.get('callback')
Results
  .Ok { s"$callback($json)" }
  .withHeader("application/javascript")
\end{lstlisting}
      \end{column}
      \begin{column}{.5\textwidth}
\begin{lstlisting}[style=html, caption={HTML}]
<script>
function callback(json) {
  /* サーバーのJSONをつかった処理 */
}
</script>
<script src="https://server.example.com/?callback=callback">
\end{lstlisting}        
      \end{column}
    \end{columns}

    \item すると\lstinline|callback( {'json': ...} )|というようなJSがロードされる
    \begin{itemize}
      \item \lstinline|{'json': ...}|はサーバーが作ったデータ
    \end{itemize}

    \pause
    \item サーバーが作ったJSONをJSの関数に入れて実行できる
  \end{itemize}
\end{frame}

\begin{frame}
  \frametitle{JSONP \textit{vs} CORS}

  \simplecallout[{\includegraphics[width=1.2cm]{./img/hacker_white.png}}]{+}{cyan!10}{%
    JSONPでクロスサイト通信は解決!なぜCORS(?)があるの?\ce{:thinking:}%
  }

  \pause
  \simplecallout[{\includegraphics[width=1cm]{./img/hacker_laugh.png}}]{-}{red!10}{%
    JSONPはXSS脆弱性の踏み台になる\ce{:smiling_imp:} \\
    (この話は\emph{Content Security Policy}などが絡み高度なので割愛)%
  }

  \pause
  \simplecallout[{\includegraphics[width=1.3cm]{./img/seminar.png}}]{+}{green!10}{%
    そこでより洗練された方法がCORS!%
  }
\end{frame}

\section{CORS(Cross-Origin Request Sharing)}

\begin{frame}[fragile]
  \frametitle{CORS(Cross-Origin Request Sharing)}

  \pause
  \begin{columns}
    \begin{column}{.55\textwidth}
      \begin{minipage}{\textwidth}
        \begin{plantuml}
          @startuml
          
          participant Website AS "Webサイト"
          database Server AS "サーバー"
       
          autonumber
          Website -> Server: このOriginならOKですか?
          Server -> Server: OriginがOKか検査
          alt OK
            Server --> Website: OKです(CORSヘッダー返す)
          else NG
            Server --> Website: NGです
            note right
              SOPによりサイトを
              跨いだ通信はブロック!
            end note
          end
          @enduml
        \end{plantuml}
      \end{minipage}
    \end{column}
    \begin{column}{.45\textwidth}        
      \begin{itemize}
        \item \ce{:point_left:}こんな感じで事前にOKなオリジンかどうかをサーバーに問合せる

        \item OKなときサーバーが専用のCORSのヘッダーを返すと、ブラウザーがそれを検出して
        SOPによる制限を緩和してくれる
      \end{itemize}
    \end{column}
  \end{columns}

  \begin{itemize}
    \item この\lstinline|Origin|ヘッダーはブラウザーが付けるので攻撃者が改竄することはできない
  \end{itemize}
\end{frame}

\begin{frame}[fragile]
  \frametitle{CSRFトークン}

  \begin{itemize}
    \item 冒頭で紹介したCSRFトークンはXHRやFetch APIではない\lstinline|form|タグ
    などで利用する
    \begin{itemize}
      \item このときにクロスドメインならCORSを使う
    \end{itemize}

    \item ブラウザーはXHRなどにしか\lstinline|Origin|ヘッダーを付与しない
  \end{itemize}

  \begin{center}
    \begin{minipage}{.4\textwidth}
      \begin{plantuml}
        @startuml
        
        participant Website AS "Webサイト"
        database Server AS "サーバー"
     
        autonumber
        Website -> Server: CSRFトークンを要求する
        Server --> Website: CSRFトークンを返す
        Website -> Server: CookieとCSRFトークン
        Server -> Server: 処理
        @enduml
      \end{plantuml}
    \end{minipage}
  \end{center}
\end{frame}

\section{まとめ}

\begin{frame}
  \frametitle{まとめ}
  
  \begin{itemize}
    \item CORSはCSRFの理解が前提となるし、かつJSONPがなぜ死んだのかも知っておかないといけないから難しい

    \item JSONPがXSSの踏み台になるなど、CSRFとXSSは実は微妙に関係している
    \begin{itemize}
      \item とはいえ基本的には別ものな脆弱性
    \end{itemize}

    \item CSRFはXSSより自由度が低いけど、サーバーのAPI呼び出しによって可能な
    任意の操作ができてしまう危険な脆弱性
    \begin{itemize}
      \item 任意コード実行のような雰囲気で考えている
    \end{itemize}

    \item XSSについては別の機会で紹介したい
  \end{itemize}
\end{frame}

\section*{参考文献}

\begin{frame}%[allowframebreaks]
  \frametitle{参考文献}

  \nocite{*}
  \bibliographystyle{junsrt_url}
  \bibliography{ref}
\end{frame}
  
\begin{frame}
  \centering
  {\Huge Thank you for the attention!}
\end{frame}

\end{document}
